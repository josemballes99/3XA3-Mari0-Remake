\documentclass{article}

\usepackage{booktabs}
\usepackage{tabularx}

\title{SE 3XA3: Development Plan\\Mari0}

\author{Team 9, Ninetendo
		\\ David Hobson hobsondd
		\\ Jose Miguel Ballesteros ballesjm
		\\ Jeff Pineda pinedaj
}

\date{}

\input{../Comments}

\begin{document}

\begin{table}[hp]
\caption{Revision History} \label{TblRevisionHistory}
\begin{tabularx}{\textwidth}{llX}
\toprule
\textbf{Date} & \textbf{Developer(s)} & \textbf{Change}\\
\midrule
Date1 & Name(s) & Description of changes\\
Date2 & Name(s) & Description of changes\\
... & ... & ...\\
\bottomrule
\end{tabularx}
\end{table}

\newpage

\maketitle

Put your introductory blurb here.

\section{Team Meeting Plan}
Our team will be meeting at the regular lab times to quickly discuss changes as well as upcoming deadlines for the project. Our lab times are Wednesday from 8:30 - 10:20 and Friday 2:30 - 4:20. If there are other times that we need to meet, we will discuss using the Facebook chat to decide on a time where all team members are available. 

\section{Team Communication Plan}
The communication will mainly happen through a Facebook chat that all team members have access to. Everyone apart of the chat can communicate quickly and effectively about milestones, changes, and meetings. Any shared files will be pushed to the repository on GitLab that everyone in the group has access to. E-mail may be used as well to send certain files or messages but this will be used rarely. 

\section{Team Member Roles}
\textbf{Project Leader and Developer:}  David Hobson\\
\textbf{Developer and Graphic Designer:}  Jose Miguel Ballesteros\\
\textbf{Developer and Documentation Manager:}  Jeff Pineda\\

\section{Git Workflow Plan}

\section{Proof of Concept Demonstration Plan}

\section{Technology}

\section{Coding Style}
We will be using Microsoft?s coding convention for C\#(INSERT REFERENCE TO LINK HERE) with a couple alterations.
- Open curly braces will follow the expression?s parameters and the closing brace will be inline with the expression AS SHOWN IN IMAGE BELLOW
- All constants will be named in capital letters
- Method names will begin with a lowercase letter, for the following words will begin with a capital letter

\section{Project Schedule}

Provide a pointer to your Gantt Chart.

\section{Project Review}

\end{document}
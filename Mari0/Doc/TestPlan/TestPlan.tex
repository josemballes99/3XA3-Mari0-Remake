\documentclass[12pt, titlepage]{article}

\usepackage{booktabs}
\usepackage{tabularx}
\usepackage{hyperref}
\usepackage{indentfirst}

\hypersetup{
    colorlinks,
    citecolor=black,
    filecolor=black,
    linkcolor=red,
    urlcolor=blue
}
\usepackage[round]{natbib}

\title{SE 3XA3: Test Plan\\Mari0}

\author{Team 9, Ninetendo
		\\ David Hobson - hobsondd
		\\ Jose Miguel Ballesteros - ballesjm
		\\ Jeff Pineda - pinedaj
}

\date{\today}

%% Comments

\usepackage{color}

\newif\ifcomments\commentstrue

\ifcomments
\newcommand{\authornote}[3]{\textcolor{#1}{[#3 ---#2]}}
\newcommand{\todo}[1]{\textcolor{red}{[TODO: #1]}}
\else
\newcommand{\authornote}[3]{}
\newcommand{\todo}[1]{}
\fi

\newcommand{\wss}[1]{\authornote{blue}{SS}{#1}}
\newcommand{\ds}[1]{\authornote{red}{DS}{#1}}
\newcommand{\mj}[1]{\authornote{red}{MSN}{#1}}
\newcommand{\cm}[1]{\authornote{red}{CM}{#1}}
\newcommand{\mh}[1]{\authornote{red}{MH}{#1}}

% team members should be added for each team, like the following
% all comments left by the TAs or the instructor should be addressed
% by a corresponding comment from the Team

\newcommand{\tm}[1]{\authornote{magenta}{Team}{#1}}


\begin{document}

\maketitle

\pagenumbering{roman}
\tableofcontents
\listoftables
\listoffigures

\begin{table}[bp]
\caption{\bf Revision History}
\begin{tabularx}{\textwidth}{p{3cm}p{2cm}X}
\toprule {\bf Date} & {\bf Version} & {\bf Notes}\\
\midrule
2016-10-31 & 1.0 & Created document\\
Date 2 & 1.1 & Notes\\
\bottomrule
\end{tabularx}
\end{table}

\newpage

\pagenumbering{arabic}

This document ...

\section{General Information}

\subsection{Purpose}
Testing is a critical part of developing software to both ensure the program meets the project's requirements and to address the errors of the product. 
\subsection{Scope}
In regard to our project, the game Mari0, the tests implemented will need to be carefully chosen in order to ensure that the game is above all else, fun. The chosen tests must also give the stakeholders of the project the assurance and confidence that the game plays and functions accordingly to the requirements outlined in the Software Requirements Specification (SRS) document. Testing will be automated where possible, however it is a necessity to test for certain attributes of the gamemanually.
\subsection{Acronyms, Abbreviations, and Symbols}
	
\begin{table}[hbp]
\caption{\textbf{Table of Abbreviations}} \label{Table}

\begin{tabularx}{\textwidth}{p{3cm}X}
\toprule
\textbf{Abbreviation} & \textbf{Definition} \\
\midrule
A.I. & Artificial Intelligence\\
G.U.I. / U.I. & Graphical User Interface / User Interface\\
SMB & Super Mario Bros.\\
SRS & Software Requirements Specifications\\
\bottomrule
\end{tabularx}

\end{table}


\begin{table}
\caption{Table of Definitions}
\begin{tabular}{l l}
\hline
Term			& Definition \\
\hline
Mario      		 & The character that the player portrays  \\
Portals      		 & Two connected portals that allow characters and projectiles to \\
			 & enter one and exit through the other, whilst mainting physial \\
			 & properties such as velocity and acceleration \\
Goomba		 & Enemy character that is defeated after the player stomps on the top \\
			 & of its head	  \\
Koopa Troopa	 & Enemy character that is defeated after the player stomps on \\
			 & the top of its head; leaves behind a shell that can be used as a projectile	\\
Lives			 & The amount of times the player can die before game over	\\
Question Block	 & Blocks found that when hit give the player coins or power ups	\\
Super Mushroom	 & Type of power up, lets Mario take an extra hit from enemies	\\
Fire Flower		 & Type of power up, gives Mario the ability to throw fireballs	\\
Time Limit		 & The amount of time in seconds the player has left complete the \\
			 & level before losing a life and restarting the level	\\
Score			 & The amount of points a player has obtained by collecting coins \\
			 & or killing enemy characters	\\
\hline
\end{tabular}
\end{table}

\subsection{Overview of Document}
	This document will describe our project of reimplementing Mari0, the test team and the tools and frameworks that they will be using, and a list and description of the tests we will be using to ensure our game meets the requirements listed in the SRS.
\section{Plan}
	
\subsection{Software Description}
The core game mechanics of Mari0 involve the simple platforming mechanics of moving left or right, and jumping; in addition to portals which allow characters and objects to be teleported between two portals while maintaining their physical characteristics.
\subsection{Test Team}
All members of Ninetendo, David Hobson, Jose Miguel Ballesteros, and Jeff Pineda, are to actively create and implement tests and test cases, and are also responsible for recording each tests' results.
\subsection{Automated Testing Approach}
If a method being written can be tested automatically, the test will be concurrently written alongside it or shortly after the method's completion. 
\subsection{Testing Tools}
As mentioned in the technology section of our Development Plan document, we will be using the NUnit framework to conduct our unit tests. Manual testing will be done in the Unity play mode interface.
\subsection{Testing Schedule}

\begin{table}[hbp]
\caption{\textbf{Testing Schedule}} \label{Table}

\begin{tabularx}{\textwidth}{p{3cm}X}
\toprule
\textbf{Date} & \textbf{What is being tested} \\
\midrule
	Wed Nov 2 &	Testing player input and character movement \\
	Fri Nov 4 & Testing character physics and character rules (death)\\
	Wed Nov 9 & Testing enemy A.I. and A.I. rules\\
	Fri Nov 11 & Testing portal mechanics\\
	Mon Nov 14 & Testing level design\\
\bottomrule
\end{tabularx}
\end{table}

\href{run:../../ProjectSchedule/ProjectSchedule.gan}{Click here for the Gantt Chart.}

\section{System Test Description}
	
\subsection{Tests for Functional Requirements}

\subsubsection{Input Testing}

\begin{enumerate}

\item{Run Right (Starting from Idle)\\}

Type: Dynamic, Manual

Initial State: In-game state where the character is not moving

Input: RIGHT\_ARROW or D\_KEY on keyboard are pressed 

Output: The character will be moving towards the right at a constant speed

Tester(s): Development Team and Colleagues

Description: The tester will make sure that the character?s movement functions as expected

\item{Run Left (Starting from Idle)\\}

Type: Dynamic, Manual

Initial State: In-game state where the character is not moving

Input: LEFT\_ARROW or A\_KEY on keyboard are pressed

Output: The character will be moving towards the left at a constant speed

Tester(s): Development Team and Colleagues

Description: The tester will make sure that the character?s movement functions as expected

\item{Run Right (Starting from moving left)\\}

Type: Dynamic, Manual

Initial State: In-game state where the character is currently moving left at a constant speed

Input: RIGHT\_ARROW or D\_KEY on keyboard are pressed

Output: The character will be moving towards the right at a constant speed

Tester(s): Development Team and Colleagues

Description: The tester will make sure that the character?s movement functions as expected

\item{Run Left (Starting from moving right)\\}

Type: Dynamic, Manual

Initial State: In-game state where the character is currently moving right at a constant speed

Input: LEFT\_ARROW or A\_KEY on keyboard are pressed

Output: The character will be moving towards the left at a constant speed

Tester(s): Development Team and Colleagues

Description: The tester will make sure that the character?s movement functions as expected

\item{Stop (Starting from moving right)\\}

Type: Dynamic, Manual

Initial State: In-game state where the character is currently moving right at a constant speed

Input: RIGHT\_ARROW or D\_KEY on keyboard are released

Output: The character will reach a state of rest

Tester(s): Development Team and Colleagues

Description: The tester will make sure that the character stops moving accordingly

\item{Stop (Starting from moving left)\\}

Type: Dynamic, Manual

Initial State: In-game state where the character is currently moving right at a constant speed

Input: LEFT\_ARROW or A\_KEY on keyboard are released

Output: The character will reach a state of rest

Tester(s): Development Team and Colleagues

Description: The tester will make sure that the character stops moving accordingly

\item{Jump (Starting from Idle)\\}

Type: Dynamic, Manual

Initial State: In-game state where the character is not moving

Input: SPACE\_BAR on keyboard is pressed

Output: The character will move upwards with an instantaneous force against a platform object it is on top of

Tester(s): Development Team and Colleagues

Description: The tester will make sure that the motion of jumping stays consistent through the trial. The tester must make sure that no continuous anti-gravitational force is applied. 

\item{Jump (Starting from running)\\}

Type: Dynamic, Manual

Initial State: In-game state where the character is currently running

Input: SPACE\_BAR on keyboard is pressed

Output: The character will move upwards with an instantaneous force against a platform object it is on top of and should contain the current horizontal speed

Tester(s): Development Team and Colleagues

Description: The tester will make sure that the motion of jumping stays consistent through the trial. The tester must make sure that no continuous anti-gravitational force is applied.

\item{Jump (While airborne)\\}

Type: Dynamic, Manual

Initial State: In-game state where the character is currently midair

Input: SPACE\_BAR on keyboard is pressed

Output: The character?s current movement will continue without alterations

Tester(s): Development Team and Colleagues

Description: The tester must make sure that no continuous anti-gravitational force is applied.

\item{Fire Blue Portal\\}

Type: Dynamic, Manual

Initial State: Any In-game state

Input: LEFT\_CLICK on supported platform

Output: Blue portal is formed on platform

Tester(s): Development Team and Colleagues

Description: The tester checks if portals are able to be fired on a platform that is at least two units long.

\item{Fire Orange Portal\\}

Type: Dynamic, Manual

Initial State: Any in-game state

Input: RIGHT\_CLICK on supported platform

Output: Orange portal is formed on platform

Tester(s): Development Team and Colleagues

Description: The tester checks if portals are able to be fired on a platform that is at least two units long.

\item{Pause\\}

Type: Dynamic, Manual

Initial State: Any In-Game State

Input: P\_KEY on keyboard is pressed

Output: The Pause menu is brought up

Tester(s): Development Team and Colleagues

Description: The tester checks to see if they can pause the current game

\item{Play Game\\}

Type: Dynamic, Manual

Initial State: Main Menu

Input: LEFT\_CLICK on PLAY option in the main menu

Output: The game begins

Tester(s): Development Team and Colleagues

Description: The tester checks to see if they can start a new game

\item{Help\\}

Type: Dynamic, Manual

Initial State: Main Menu

Input: LEFT\_CLICK on HELP option in the main menu

Output: The help menu is brought up

Tester(s): Development Team and Colleagues

Description: The tester checks to see if they can view the help menu

\item{Options\\}

Type: Dynamic, Manual

Initial State: Main Menu

Input: LEFT\_CLICK on OPTIONS option in the main menu

Output: The options menu is brought up

Tester(s): Development Team and Colleagues

Description: The tester checks to see if they can view the options menu

\end{enumerate}

\subsubsection{Object Collision Testing}

\begin{enumerate}

\item{Collision with Wall\\}

Type: Dynamic, Manual

Initial State: Character is moving toward wall

Input/Condition: Character hits wall object

Output: Character is stopped by wall

Tester(s): Development Team and Colleagues

Description: The tester checks if characters make impact with wall


\item{Collision with Floor Platform\\}

Type: Dynamic, Manual

Initial State: Character is on/falling towards floor platform

Input/Condition: Character hits floor object

Output: Character lands and stays on platform

Tester(s): Development Team and Colleagues

Description: The tester checks if floor platforms are working as expected


\item{Collision with Flag\\}

Type: Dynamic, Manual

Initial State: Character is moving toward a flag

Input/Condition: Character hits the flag object

Output: Level is won user stops controlling character

Tester(s): Development Team and Colleagues

Description: The tester checks if flag objects work and can finish the level


\item{Collision with Blue Portal\\}

Type: Dynamic, Manual

Initial State: Game Object is moving toward portal

Input/Condition: Game Object hits blue portal object

Output: Game Object teleports to the orange portal?s location conserving previous movement speed

Tester(s): Development Team and Colleagues

Description: The tester checks if Game Objects can teleport through portals


\item{Collision with Orange Portal\\}

Type: Dynamic, Manual

Initial State: Game Object is moving toward portal

Input/Condition: Game Object hits orange portal object

Output: Game Object teleports to the blue portal?s location conserving previous movement speed

Tester(s): Development Team and Colleagues

Description: The tester checks if Game Objects can teleport through portals


\item{Collision with Goombas (front facing collision)\\}

Type: Dynamic, Manual

Initial State: Character is moving toward a Goomba

Input/Condition: Character hits a Goomba body front first

Output: Character loses a life

Tester(s): Development Team and Colleagues

Description: The tester checks if Goombas can defeat the character


\item{Collision with Goombas (foot first collision)\\}

Type: Dynamic, Manual

Initial State: Character is moving toward a Goomba

Input/Condition: Character hits a Goomba body foot first
Output: Goomba is defeated

Tester(s): Development Team and Colleagues

Description: The tester checks if they can defeat the Goomba


\item{Collision with Koopas (front facing collision)\\}

Type: Dynamic, Manual

Initial State: Character is moving toward a Koopa

Input/Condition: Character hits a Koopa body front first

Output: Character loses a life

Tester(s): Development Team and Colleagues

Description: The tester checks if Koopas can defeat the character


\item{Collision with Koopas (foot first collision)\\}

Type: Dynamic, Manual

Initial State: Character is moving toward a Koopa

Input/Condition: Character hits a Koopa body foot first

Output: Koopa transforms to shell form

Tester(s): Development Team and Colleagues

Description: The tester checks if they can defeat the Koopa


\item{Collision with Koopa Shell (front facing collision)\\}

Type: Dynamic, Manual

Initial State: Character is moving toward a Koopa Shell

Input/Condition: Character hits a Koopa shell front first

Output: Character loses a life

Tester(s): Development Team and Colleagues

Description: The tester checks if Koopa Shells can defeat the character


\item{Collision with Koopa Shell (foot first collision)\\}

Type: Dynamic, Manual

Initial State: Character is moving toward a Koopa Shell

Input/Condition: Character hits a Koopa shell foot first

Output: Koopa shell is defeated

Tester(s): Development Team and Colleagues

Description: The tester checks if they can defeat the Koopa Shell



\item{Collision with Bullet Bill (front facing collision)\\}

Type: Dynamic, Manual

Initial State: Character is moving toward Bullet Bill

Input/Condition: Character hits a Bullet Bill body front first

Output: Character loses a life

Tester(s): Development Team and Colleagues

Description: The tester checks if Bullet Bill can defeat the character



\item{Collision with Bullet Bill (foot first collision)\\}

Type: Dynamic, Manual

Initial State: Character is moving toward Bullet Bill

Input/Condition: Character hits a Bullet Bill body foot first

Output: Bullet Bill is defeated

Tester(s): Development Team and Colleagues

Description: The tester checks if they can defeat a Bullet Bill

\end{enumerate}

\subsection{Tests for Nonfunctional Requirements}
The different tests listed will demonstrate that the non-functional requirements in the software requirements specification are met.

\subsubsection{Look and Feel Requirements}

\begin{enumerate}

\item{Game Environment\\}

Type: Dynamic, Manual

Initial State: In-game state

Tester(s): Development Team, Colleagues and/or testing group

Description: The tester will see if the different environments are correct and meet the specifications
					
\item{Game Hude/Interface\\}

Type: Dynamic, Manual

Initial State: In-game state

Tester(s): Development Team, Colleagues, and/or testing group

Description: The tester will make sure that the score, time, lives, and amount of coins is not obstructive in the game?s view.

\end{enumerate}

\subsubsection{Usability and Humanity Requirements}

\begin{enumerate}

\item{Ease of Learning\\}

Type: Dynamic, Manual

Initial State: In-game state

Tester(s): Development Team, Colleagues, and/or testing group

Description: The tester will play through the game and will inform the development team of clarifications

\item{Entertainment\\}

Type: Dynamic, Manual

Initial State: In-game state

Tester(s): Development Team, Colleagues, and/or testing group

Description: The tester will make sure that the game is entertaining and follows similar principles to that of the original game

\end{enumerate}

\subsubsection{Performance Requirements}

\begin{enumerate}

\item{Controls/Commands\\}

Type: Dynamic, Manual

Initial State: In-game state

Tester(s): Development Team, Colleagues, and/or testing group

Description: The tester will make sure that the game does not have any noticeable delays with controls, as well as any controls that seem odd/difficult to understand.

\end{enumerate}

\subsubsection{Operational and Environment Requirements}

\begin{enumerate}

\item{Operating System Support\\}

Type: Dynamic, Manual

Initial State: Downloading/Installing

Tester(s): Development Team, Colleagues, and/or testing group.

Description: The tester will make sure that the game is able to run on Windows, MacOS, and Ubuntu.

\end{enumerate}

\subsubsection{Security Requirements}

\begin{enumerate}

\item{Altering Information\\}
Type: Dynamic, Manual

Initial State: In-game state

Tester(s): Development Team, Colleagues, and/or testing group

Description: The tester will make sure that the game does not alter any files or processes that are not directly related to the game.

\end{enumerate}


\subsubsection{Cultural Requirements}

\begin{enumerate}

\item{Spelling and Grammar\\}

Type: Dynamic, Manual

Initial State: In-game state

Tester(s): Development Team, Colleagues, and/or testing group

Description: The tester will make sure that the game has no spelling/grammar errors and that messages, menus, and overall interface is written in English.

\item{Offensive Content\\}
Type: Dynamic, Manual

Initial State: In-game state

Tester(s): Development Team, Colleagues, and/or testing group

Description: The tester will make sure that the game has no offensive content towards culture (religion, politics, ethnics, race, etc?).

\end{enumerate}

\subsubsection{Legal Requirements}

\begin{enumerate}

\item{License Adherence\\}

Type: Dynamic, Manual

Tester(s): Development Team, and Colleagues

Description: The tester will make sure that the game is not breaching the license that comes along with the game.

\end{enumerate}

\subsubsection{Health and Safety Requirements}

\begin{enumerate}

\item{Epileptic Prevention\\}

Type: Dynamic, Manual

Initial State: In-game state

Tester(s) ? Development Team, Colleagues, and/or testing group

Description: The tester will make sure that the game does not trigger epileptic seizures as a result from playing.

\end{enumerate}

\section{Tests for Proof of Concept}

A proof on concept test will be used to show that the development for Mari0 is feasible with the current skills and technology we have available to us. This section describes the proof of concept test and the details associated with it.

\subsection{Demonstration Plan}
For a proof of concept test we will create a small prototype that will be ran from Unity that can be used on Windows 10, MacOS, and Ubuntu. The prototype will be a small game demo demonstrating collision detection with different in game objects, the main gravity system, and the portal interactions. Many of these different game elements will be implemented using the Unity's collision and physics engines, as this will make our final goal easier to achieve. The main  graphics that are used in the actual game will be used for this demo.

The prototype will be a floor that will be similar to the final game, which will be populated with the player character, six portals, two pipes, and platforms which the character can interact with. The player will be able to stand on the main floor and the platforms. There are also no walls to contain the character on either side.

The player (which will be represented by Mario) can be controlled in the following ways:
\begin{itemize}  
\item The player moves left and right with the 'a' and 'd' keys respectively
\item The player can jump by using the spacebar
\end{itemize}


The player will interact with the different objects in the following ways:
\begin{itemize}  
\item The floor will be the main platform that the user will be able to stand on.
\item The pipes will act like walls when approached from the side and not allow the user to pass through, and act like a floor when approached from the top.
\item All 6 portals are paired in different ways, when entering a blue portal, the character will exit the orange portal, and vice versa.
\item All physics will be maintained when entering through a portal, and portals can be on any surface that is at least two units.
\end{itemize}


\paragraph{Proof of Concept Test}

Many of the tests that are demonstrated in the proof of concept will be stated in the System Tests section of this document.

\begin{enumerate}

\item{Proof of Concept}

Type: Manual

Tester(s): Development Team and Colleagues

Description: Tests whether significant risks to the completion of the project can be overcome.

\end{enumerate}
	
\section{Comparison to Existing Implementation}	
	Since the functional and non-function requirements that Mari0 must meet are derived from the original implementation, all system tests that tests whether or not these requirements are met will be the metric used to determine the likeness of our reimplementation to the original product.
\section{Unit Testing Plan}
	The NUnit framework will be used to conduct all unit tests of Mari0. We will conduct our testing in a manner that satisfies a complete condition coverage criteria.
\subsection{Unit testing of internal functions}
	Internal functions and methods that return any value type, such as integer, float, boolean, etc. can and will be tested dynamically, with player inputs or input conditions being given and comparing the actual output to the expected output. Due to the development and testing schedule, the use of drivers is not needed, and the use of stubs will be very limited if they are used at all. Since any video game is inherently dependant on player actions and how their interactions affect the game, we will be testing for complete condition coverage.
\subsection{Unit testing of output files}		
	Mari0 will have no output files to be tested. The only output file the original implementation had was a saved game file that kept track of the players progress, however, due to limiting our reimplementation of the game to a single level, a save file is unnecessary.
\bibliographystyle{plainnat}

\bibliography{SRS}

\newpage

\section{Appendix}

This is where you can place additional information.

\subsection{Symbolic Parameters}

The definition of the test cases will call for SYMBOLIC\_CONSTANTS.
Their values are defined in this section for easy maintenance.

\subsection{Usability Survey Questions}

\begin{itemize}
Since many functionalities of Mari0 are difficult to test, either due to testing having to be done manually or a bias by the development team during testing, the following usability suvey questions have been listed to test some of these difficult functionalities
  \item Are the portal mechanics intuitive and easy to understand?
  \item Do the controls feel smooth?
  \item Do you feel like you have full or good control of the character?
  \item Are the U.I. elements obtrusive? Easily readable? Aesthetically pleasing?
  \item Do you think the game is fair?
  \item Do you think the game is fun?
\end{itemize}

\end{document}
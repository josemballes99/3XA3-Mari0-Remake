\documentclass[12pt, titlepage]{article}

\usepackage{booktabs}
\usepackage{tabularx}
\usepackage{hyperref}
\hypersetup{
    colorlinks,
    citecolor=black,
    filecolor=black,
    linkcolor=red,
    urlcolor=blue
}
\usepackage[round]{natbib}

\title{SE 3XA3: Development Plan\\Mari0}

\author{Team 9, Ninetendo
		\\ David Hobson - hobsondd
		\\ Jose Miguel Ballesteros - ballesjm
		\\ Jeff Pineda - pinedaj
}

\date{\today}

%% Comments

\usepackage{color}

\newif\ifcomments\commentstrue

\ifcomments
\newcommand{\authornote}[3]{\textcolor{#1}{[#3 ---#2]}}
\newcommand{\todo}[1]{\textcolor{red}{[TODO: #1]}}
\else
\newcommand{\authornote}[3]{}
\newcommand{\todo}[1]{}
\fi

\newcommand{\wss}[1]{\authornote{blue}{SS}{#1}}
\newcommand{\ds}[1]{\authornote{red}{DS}{#1}}
\newcommand{\mj}[1]{\authornote{red}{MSN}{#1}}
\newcommand{\cm}[1]{\authornote{red}{CM}{#1}}
\newcommand{\mh}[1]{\authornote{red}{MH}{#1}}

% team members should be added for each team, like the following
% all comments left by the TAs or the instructor should be addressed
% by a corresponding comment from the Team

\newcommand{\tm}[1]{\authornote{magenta}{Team}{#1}}


\begin{document}

\maketitle

\pagenumbering{roman}
\tableofcontents
\listoftables
\listoffigures

\begin{table}[bp]
\caption{\bf Revision History}
\begin{tabularx}{\textwidth}{p{3cm}p{2cm}X}
\toprule {\bf Date} & {\bf Version} & {\bf Notes}\\
\midrule
October 6, 2016 & 1.0 & Created Document, rough draft of section 1\\
Date 2 & 1.1 & Notes\\
\bottomrule
\end{tabularx}
\end{table}

\newpage

\pagenumbering{arabic}

This document describes the requirements for Mari0  The template for the Software
Requirements Specification (SRS) is a subset of the Volere
template~\citep{RobertsonAndRobertson2012}.

\section{Project Drivers}

\subsection{The Purpose of the Project}
The purpose of this project is to recreate the game, Mari0, to allow the players 
to entertain themselves and alleviate their boredom. Mari0 is a combination of
Super Mario Bros. and Portal, challenging a player's platforming abilities and their
puzzle solving skills.

\subsection{The Stakeholders}

\subsubsection{The Client}
The client for the Mari0 project is the game publisher.

\subsubsection{The Customers}
The customers for this project are people interested in platforming and/or puzzle games.

\subsubsection{Other Stakeholders}
The other stakeholders for Mari0 are the game's developers and designers.

\subsection{Mandated Constraints}
The constraints as mandated by our client are as follows:
\begin{itemize}
	\item Have each deliverable finished by the deadlines given in the course outline
	\item The game's physics will be handled by the Unity Game engine
	\item The product should be runnable on all operate systems
\end{itemize}

\subsection{Naming Conventions and Terminology}

\begin{tabular}{l l}
\hline
Term			& Definition \\
\hline
A.I.			 & Artificial Intelligence  \\
Portals      		 & Two connected portals that allow characters and projectiles to \\
			 & enter one and exit through the other, whilst mainting physial \\
			 & properties such as velocity and acceleration \\
Mario      		 & The character that the player portrays  \\
Goomba		 & Enemy character that is defeated after the player stomps on the top \\
			 & of its head	  \\
Lives			 & The amount of times the player can die before game over	\\
Question Block	 & Blocks found that when hit give the player coins or power ups	\\
Fire Flower		 & Type of power up, gives Mario the ability to throw fireballs	\\
Super Mushroom	 & Type of power up, lets Mario take an extra hit from enemies	\\
Koopa Troopa	 & Enemy character that is defeated after the player stomps on \\
			 & the top of its head; leaves behind a shell that can be used as a projectile	\\
\hline

\end{tabular}

\subsection{Relevant Facts and Assumptions}

User characteristics should go under assumptions.

\section{Functional Requirements}

\subsection{The Scope of the Work and the Product}

\subsubsection{The Context of the Work}

\subsubsection{Work Partitioning}

\subsubsection{Individual Product Use Cases}

\subsection{Functional Requirements}

\section{Non-functional Requirements}

\subsection{Look and Feel Requirements}

\subsection{Usability and Humanity Requirements}

\subsection{Performance Requirements}

\subsection{Operational and Environmental Requirements}

\subsection{Maintainability and Support Requirements}

\subsection{Security Requirements}

\subsection{Cultural Requirements}

\subsection{Legal Requirements}

\subsection{Health and Safety Requirements}

This section is not in the original Volere template, but health and safety are
issues that should be considered for every engineering project.

\section{Project Issues}

\subsection{Open Issues}
Currently, there a no known major issues for the project, however as the implementation continues there may need to be changes and problems may start to occur. 
Here are the list of some current issues with the game that may need to be improved:
\begin{itemize}
\item Some old operating systems such as Windows XP and Windows 7 have difficulty running the game properly.
\item Lack of modding support.
\item Players run into problems accessing the save folders.
\item Players have trouble using the Love framework in order to run the game
\end{itemize}

\subsection{Off-the-Shelf Solutions}
A lot of the issues will be solved easily with the use of Unity, since it is professionally made for game developers, and it is accessible to many operating systems. Also players will not have to deal with the Love framework that is apart of the current Mari0 game, since the entire project will be created in Unity.
Furthermore, if there are any huge problems, the code that has created the current game can be slightly modified to fit with Unity and will be essentially used as a prototype. In addition, Unity has it's own physics engine that will be a huge help in creating the environment for the game, instead of creating the entire game environment from scratch.
It may also help to look into other successful Unity games to see how things have been implemented, some platforming games may give us insight into how to make sure the physics works best for our final product. 

\subsection{New Problems}
New problems have yet to arise in the implementation. However, using Unity may add more problems in how things will be ran on certain systems if we would like to port the game to a mobile device or a console system. Overall, no new problems are expected to rise as a result of this project.

\subsection{Tasks}
Tasks are listed and numbered below.
\begin{enumerate}
\item Structures - Create class hierarchies and main game objects.
\item Overall Mechanics - Getting the character moving between two portals and interacting with environment
\item Level Design - Creating levels that the user can play and the character can be placed into.
\item Interfaces - Main programming interfaces such as Menu, Game Over, and Pause Screens.
\item Graphics and Sound - Main graphics, music and animation for the game.
\item Improvements - Adding different aspects of the game such as newer levels or different mechanics.
\end{enumerate}

\subsection{Migration to the New Product}
Since we are creating the same product, there is no migration to a newer product at this time.

\subsection{Risks}
Overall, risks are few and far between when it comes to recreating Mari0, but there are some risks that we would like to minimize. Flashing colours on the screen may trigger epileptic seizures for some users. Also, if the game is not optimized well, overheating of the system may damage users systems or cause minor burns. Although the chances of these problems arising are extremely low, they will be kept in mind when creating the final product.

\subsection{Costs}
Currently there are no costs associated with this project.

\subsection{User Documentation and Training}
User Documentation will be created as per the SFWR 3XA3 guidelines. Training/Tutorial will be implemented into the game through screen shots or a small in game user manual.

\subsection{Waiting Room}
There are currently no requirements or problems that have not been met or solved. This section will be updated as needed.

\subsection{Ideas for Solutions}
There are currently no ideas for solutions and no overall plan for these solutions. This section will be updated as needed.

\bibliographystyle{plainnat}

\bibliography{SRS}

\newpage

\section{Appendix}

This section has been added to the Volere template.  This is where you can place
additional information.

\subsection{Symbolic Parameters}

The definition of the requirements will likely call for SYMBOLIC\_CONSTANTS.
Their values are defined in this section for easy maintenance.


\end{document}